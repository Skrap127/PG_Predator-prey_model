\chapter{Preliminares}
    Para el análisis del módelo presa-depredador tipo Leslie-Gower con respuesta funcional sigmoidea que se lleva acabo en el presente trabajo, requerimos de ciertos conceptos fundamentales, los cuales se abarcan en la presente sección, esto, con el objetivo de lograr una mejor compresión de todo lo que se presenta posteriormente.
    
    \section{Cálculo diferencial e integral}
    \subsection{Límites y continuidad}
    \subsection{Derivadas parciales}
    \subsection{Matriz Jacobiana}
    \subsection{Gradiente y campos vectoriales}
    \subsection{Series de Taylor (expansión local)}
    
    \section{Álgebra lineal}
    \subsection{Espacios vectoriales}
    \subsection{Sistemas lineales y no lineales de ecuaciones}
    \subsection{Cambios de base y transformaciones lineales}
    \subsection{Autovalores y autovectores}
    \subsection{Diagonalización de matrices}
    
    \section{Ecuaciones diferenciales ordinarias (EDOs)}
    Una ecuación difrencial ordinaria o EDO es una ecuación para una función desconocida de una sola variable real, que no solo contiene a la función sino también a sus derivadas. De manera general una EDO es de la forma
    \begin{equation*}
    	F(t, x, x^{(1)}, \cdots, x^{(n)}) = 0,
    	\label{eq: generalFormEDO}
    \end{equation*}
    donde $F$ es una función real de $n+2$ variables y $x = x(t)$ es una función desconocida de una variable real $t$. El máximo orden $n$ de la derivada $x^{(n)}$ es llamado el \textit{orden} de la ecuación diferencial.
    \teo{\textbf{Teorema de existencia y unicidad de Picard-Lindelöf}}
    \subsection{Métodos numéricos básicos (Euler, Runge-Kutta)}
    
    \section{Topología y análisis}
    \subsection{Conjuntos abiertos y cerrados}
    \subsection{Conjuntos límite}
    \subsection{Región compacta}
    \subsection{Espacios métricos}
    \subsection{Continuidad y compacidad}
    \subsection{Conjuntos invariantes}
    \subsection{Variedades}
    \subsection{Condición de transversabilidad}
    \subsection{Reparametrización de sistemas}
    
    \section{Geometría diferencial}
    \subsection{Variedades diferenciables}
    \subsection{Difeomorfismos}
    \subsection{Campos vectoriales sobre variedades}
    
    \section{Teoría cualitativa de EDOs}
    \subsection{Campos vectoriales en el plano}
    \subsection{Líneas de flujo y trayectorias}
    \subsection{Diagramas de fase}
    \subsection{Isoclinas}
    \subsection{Puntos críticos y clasificación}
    \subsection{Campo vectorial asociado a un sistema}
    \subsection{Trayectorias, órbitas y curvas solución}
    \subsection{Regiones invariantes}
    \subsection{Separatrices}
    \subsection{Compactificación de Poincaré}
    \subsection{Variedades estables e inestables}
    \subsection{Reescalamiento y reparametrización temporal}
    \subsection{Desingularización (incluyendo blowing-up)}
    
    \section{Sistemas dinámicos}
    \subsection{Sistemas autónomos y no autónomos}
    \subsection{Sistema autónomo bidimensional}
    \subsection{Sistemas equivalentes topológicamente}
    \subsection{Ciclos límite}
    \subsection{Bifurcaciones (saddle-node, Hopf, pitchfork, etc.)}
    \subsection{Diagramas de bifurcación}
    \subsection{Estabilidad de sistemas no lineales}
    \subsection{Funciones de Lyapunov}
    \subsection{Método del número de Lyapunov}
    \subsection{Órbitas heteroclínicas y homoclínicas}
    \subsection{Teorema de Poincaré Bendixson}
    
    \section{Biología matemática / Modelos ecológicos}
    \subsection{Crecimiento logístico}
    \subsection{Ecuaciones logísticas y crecimiento poblacional}
    \subsection{Modelos presa-depredador clásicos (Lotka-Volterra)}
    \subsection{Respuestas funcionales de Holling (tipos I, II, III)}
    \subsection{Modelo Leslie-Gower}
    \subsection{Modelo May–Holling–Turner}
    \subsection{Efecto Allee}
    \subsection{Ecuaciones de tipo Kolmogorov}
    
    \section{Métodos algebraicos y analíticos}
    \subsection{Regla de los signos de Descartes}
    \subsection{Análisis de estabilidad lineal y no lineal}
    \subsection{Sistemas tangentes}
    \subsection{Polar blowing-up method}
    