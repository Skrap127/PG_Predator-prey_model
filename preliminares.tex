\chapter{Preliminares}

Para el análisis del módelo presa-depredador tipo Leslie-Gower con respuesta funcional sigmoidea que se lleva acabo en el presente trabajo, requerimos de ciertos conceptos fundamentales, los cuales se abarcan en la presente sección, esto, con el objetivo de lograr una mejor compresión de todo lo que se presenta posteriormente.

\section*{Posible lista de conceptos a definir}

Conceptos relacionados a los mostrados en \cite{articulo}

\begin{enumerate}
	\item \textbf{Álgebra lineal}
	\begin{itemize}
		\item Espacios vectoriales
		\item Autovalores y autovectores \checkmark
		\item Diagonalización de matrices \checkmark
		\item Sistemas lineales y no lineales de ecuaciones \checkmark
		\item Cambios de base y transformaciones lineales
	\end{itemize}
	
	\item \textbf{Ecuaciones diferenciales ordinarias (EDOs)}
	\begin{itemize}
		\item Ecuaciones diferenciales ordinarias (EDO) \checkmark
		\item Teorema de existencia y unicidad \checkmark
		\item Métodos numéricos básicos (Euler, Runge-Kutta)
	\end{itemize}
	
	
	\item \textbf{Geometría diferencial}
	\begin{itemize}
		\item Variedades diferenciables
		\item Difeomorfismos
		\item Homeomorfismos
		\item Campos vectoriales sobre variedades
	\end{itemize}
	
	\item \textbf{Teoría cualitativa de EDOs}
	\begin{itemize}
		\item Campo vectorial asociado a un sistema \checkmark
		\item Trayectorias \checkmark
		\item Diagramas de fase \checkmark
		\item Isoclinas \checkmark
		\item Puntos críticos y clasificación \checkmark (falta ptos. centro)
		\item Conjuntos invariantes \checkmark
		\item Separatrices
		\item Compactificación de Poincaré
		\item Variedades estables e inestables
		\item Reescalamiento y reparametrización temporal
		\item Desingularización (incluyendo blowing-up)
	\end{itemize}
	
	\item \textbf{Sistemas dinámicos}
	\begin{itemize}
		\item Sistemas autónomos y no autónomos \checkmark
		\item Sistemas equivalentes topológicamente \checkmark
		\item Ciclos límite \checkmark
		\item Bifurcaciones (saddle-node, Hopf, pitchfork, etc.)
		\item Diagramas de bifurcación
		\item Estabilidad de sistemas no lineales
		\item Teorema de Hartman-Grobman \checkmark
		\item Funciones de Lyapunov
		\item Método del número de Lyapunov
		\item Órbitas heteroclínicas y homoclínicas
		\item Teorema de Poincaré-Bendixson
	\end{itemize}
	
	\item \textbf{Biología matemática / Modelos ecológicos}
	\begin{itemize}
		\item Ecuaciones logísticas y crecimiento poblacional
		\item Modelos presa-depredador clásicos (Lotka-Volterra)
		\item Respuestas funcionales de Holling (tipos I, II, III)
		\item Modelo Leslie-Gower
		\item Modelo May–Holling–Turner
		\item Crecimiento logístico
		\item Efecto Allee
		\item Ecuaciones de tipo Kolmogorov
	\end{itemize}
	
	\item \textbf{Métodos algebraicos y analíticos}
	\begin{itemize}
		\item Regla de los signos de Descartes
		\item Análisis de estabilidad lineal y no lineal
		\item Polar blowing-up method
		\item Sistemas tangentes
	\end{itemize}
\end{enumerate}

\section{Ecuaciones diferenciales ordinarias (EDOs)}

Sean $U \subseteq \Rm, V \subseteq \Rn$ y $k\in \N_{0}$. Entonces $C^{k}(U, V)$ denota el conjunto de funciones de $U \longrightarrow V$ continuamente diferenciables hasta el orden k. Adicionalmente, para simplificar denotaremos $C^{k}(U, \R)$ como $C^{k}(U)$. Una \textit{ecuación difrencial ordinaria} o EDO es una ecuación para una función desconocida de una sola variable real, que no solo contiene a la función sino también a sus derivadas. De manera general una EDO es de la forma

\begin{equation}
	F(t, x, x^{(1)}, \cdots, x^{(k)}) = 0,
	\label{eq: generalFormEDO}
\end{equation}

donde $F \in C(U)$ con $U$ un subconjunto abierto de $\R^{k+2}$, $x = x(t) \subseteq C(J)$ con $J \subseteq \R$ y

\begin{equation*}
	x^{(k)}(t) = \frac{d^{k}x(t)}{dt^{k}}, \quad k \in \N_{0}.
\end{equation*}

Al máximo orden $k$ de la derivada $x^{(k)}$ en \eqref{eq: generalFormEDO} se le llama el \textit{orden} de la ecuación diferencial. Frecuentemente a $t$ se le conoce como la \textit{variable independiente} y a $x$ como la \textit{variable dependiente}.

Una \textit{solución} de la EDO \eqref{eq: generalFormEDO}, es una función $\phi \in C^{k}(I)$ con $I \subseteq J$ un intervalo real, tal que

\begin{equation*}
	F(t, \phi(t), \phi^{(1)}(t), \cdots, \phi^{(k)}(t)) = 0, \quad \text{para todo } t \in I.
\end{equation*}

Un \textit{sistema no lineal} de EDOs \textit{no autónomo} de primer orden es un sistema de la forma

\begin{equation}
	\dot{x} = f(x, t),
	\label{eq:generalFormSisEDOs}
\end{equation}

donde $f:E \longrightarrow \Rn$ con $E$ un subconjunto abierto de $\R^{n+1}$, $x = (x_{1}, \cdots, x_{n})$ y

\begin{equation*}
	\dot{x} = \frac{dx}{dt} = 
	\begin{bmatrix}
		\frac{dx_{1}}{dt} \\
		\vdots \\
		\frac{dx_{n}}{dt}
	\end{bmatrix}.
\end{equation*}

Si la función $f$ en \eqref{eq:generalFormSisEDOs} no depende de $t$ entonces $f:\tilde{E} \longrightarrow \Rn$ con $\tilde{E}$ un subconjunto abierto de $\Rn$ y al sistema 

\begin{equation}
	\dot{x} = f(x),
	\label{eq: sisAut}
\end{equation} 

se le conoce como un \textit{sistema no lineal autónomo}.

Un \textit{sistema lineal no autónomo} es un sistema de la forma

\begin{equation*}
	\dot{x} = A(t)x + b(t),
	\label{eq: sistemaLineal}
\end{equation*}

donde $A(t)$ es una matriz $n \times n$ y $b(t)$ es un vector de $\Rn$. Si $b(t) = 0$ el sistema es \textit{homogeneo}, además si $A$ es una matriz diagonal decimos que el sistema es \textit{desacoplado} y sino \textit{acoplado}, si $b(t) \neq 0$ el sistema es \textit{no homogeneo}. En algunas ocasiones todos los anteriores tipos de sistemas pueden ser analizados a través de un sistema de la forma

\begin{equation}
	\dot{x} = Ax.
	\label{eq: sisLinAuto}
\end{equation}

\begin{teo}[Existencia y unicidad]\label{teo: Existencia y unicidad}
	Sea $E$ un subconjunto abierto de $\Rn$ que contiene a $x_{0}$ y asuma que $f \in C^{1}(E)$. Entonces existe un $a>0$ tal que el problema de valor inicial 
	\begin{equation}
		\left\{
		\begin{aligned}
			\dot{x} = f(x) \\
			x(0) = x_{0}
		\end{aligned}
		\right.
		\label{eq: pvi}
	\end{equation}
	tiene una única solución en el intervalo $[-a,a]$.
\end{teo}

Sea $J=(\alpha, \beta)$ la unión de todos los intervalos abiertos $I$ tales que \eqref{eq: pvi} tiene una solución en $I$, llamamos a J el \textit{intervalo maximal de existencia} del PVI \eqref{eq: pvi}.

\begin{defi}
	Sean $E$ un subconjunto abierto de $\Rn$, $f \in C^{1}(E)$, $x_{0} \in E$ y $\phi_{t}(x_{0})$ la solución del PVI \eqref{eq: pvi} en el intervalo maximal de existencia $I(x_{0})$. Entonces para $t\in I(x_{0})$, el conjunto de funciones $\phi_{t}$ defindas por $\phi_{t}(x_{0}) = \phi(t, x_{0})$ es llamado el flujo de la ecuación diferencial \eqref{eq: sisAut} o también el flujo del campo vectorial $f(x)$.
\end{defi}

El sistema \eqref{eq: sisAut} puede considerarse como un campo vectorial de $\Rn$ y las soluciones del sistema son curvas en $E$ que son tangentes a este campo vectorial en cada punto. Así para obtener una idea geométrica de las soluciones se puede graficar el \textit{campo vectorial asociado al sistema}. En particular, las soluciones del sistema \eqref{eq: sisAut} también se denominan \textit{curvas solución}, \textit{trayectorias} u \textit{órbitas}. Esto, en el sentido cualitativo.

El \textit{retrato fase} de un sistema de EDOs es el conjunto de todas las curvas solución del sistema \eqref{eq: pvi} en el plano $\Rn$.

En un \textit{sistema autónomo bidimensional}
\begin{equation}
	\begin{aligned}
		\dot{x} &= f(x, y) \\
		\dot{y} &= g(x, y),
	\end{aligned}
	\label{eq: sisAutBid}
\end{equation}

podemos encontrar una aproximación global de curvas solución, a través del método de las isoclinas. Del sistema \eqref{eq: sisAutBid} obtenemos el sistema de primer orden $dy/dx = g(x, y)/f(x, y)$. Ignorando el hecho de que esto podría no estar bien definido en $f(x, y) = 0$, el método consiste en encontrar curvas $y=h(x)$ o $x=h(y)$ en las que la pendiente del campo vectorial $dy/dx = c$ es constante. Dichas curvas se obtienen solucionando la ecuación

\begin{equation*}
	g(x, y) = cf(x, y),
	\label{eq: pendiente}
\end{equation*}

y son llamadas \textit{isoclinas}. Las isoclinas se pueden encontrar en sistemas de mayor dimensión pero solo es relevante en los de dos y tres dimensiones, pues su importancia radica en la interpretación gráfica de esta. Se muestra un ejemplo de todo lo anterior en la \autoref{fig: example}.

\begin{figure}
	\centering
	\includegraphics[width=0.7\textwidth]{img/Example.pdf}
	\caption{Ejemplo de campo vectorial, trayectorias, retrato fase e isoclinas ($c=0$) del sistema \eqref{eq: pvi} asociado a $f(x)=(x_{1},\sin{x_{2}})$.}
	\label{fig: example}
\end{figure}

En algunas ocasiones las trayectorias o curvas solución de un sistema pueden ser cerradas y aisladas, es decir, no hay más trayectorias cerradas en una cierta región, dichas curvas solución son denominadas un \textit{ciclo límite}; si todas las trayectorias vecinas se acercan al ciclo límite decimos que el ciclo límite es \textit{estable} o \textit{atractor}, en otros casos el ciclo límite es \textit{inestable} o en casos excepcionales \textit{medio-estable}, donde las trayectorias vecinas se alejan y se acercan simultáneamente dentro y fuera del ciclo límite. En la \autoref{fig: CicloLimiteEstableInestable} se muestra un ejemplo de estos dos primeros casos.

\begin{figure}
	\centering
	\includegraphics[width=0.7\textwidth]{img/CicloLimEstInest.pdf}
	\caption{Ejemplo de ciclos limites del sistema \eqref{eq: pvi} asociado a $f(x) = (-x_{2} + x_{1}(r^{4} -3r^{2} + 1), x_{1} + x_{2}(r^{4} -3r^{2} + 1))$ con $r^{2} = x_{1}^{2} + x_{2}^{2}$.}
	\label{fig: CicloLimiteEstableInestable}
\end{figure}

\begin{defi}
	Sea $\mathcal{M}_{n}(\R)$ el conjunto de matrices cuadradas de tamaño $n$ con entradas en $\R$ y sea $A \in \mathcal{M}_{n}(\R)$. Un escalar $\lambda \in \R$ es un valor propio de $A$ si existe un vector no nulo $v \in \Rn$ tal que, $Av = \lambda v$. Un vector $v \in \Rn$ tal que $Av = \lambda v$ es llamado un vector propio de $A$ asociado al valor propio $\lambda$.
\end{defi}

\begin{defi}
	Sea $A \in \mathcal{M}_{n}(\R)$ decimos que $A$ es diagonalizable si existe una matriz invertible $P \in \mathcal{M}_{n}(\R)$ y una matriz diagonal $D$ tal que $P^{-1}AP = D$.
\end{defi}

\begin{defi}
	Un punto $x_{0} \in \Rn$ es llamado un punto de equilibrio o punto crítico de \eqref{eq: sisAut} si $f(x_{0})=0$. Un punto crítico $x_{0}$ es llamado un punto de equilibrio hiperbólico de \eqref{eq: sisAut} si ninguno de los valores propios de la matriz jacobiana $Df(x_{0})$ tiene parte real cero. El sistema lineal \eqref{eq: sisLinAuto} con la matriz $A = Df(x_{0})$ es llamado la linealización de \eqref{eq: sisAut} en el punto $x_{0}$.
\end{defi}

\begin{defi}
	Un punto $x_{0} \in \Rn$ es llamado un sumidero si todos los valores propios de la matriz jacobiana $Df(x_{0})$ tienen parte real negativa; es llamado una fuente si todos los valores propios de la matriz $Df(x_{0})$ tienen parte real positiva; y es llamado un punto silla si es un punto de equilibrio hiperbólico y $Df(x_{0})$ tiene al menos un valor propio con parte real positiva y al menos un valor propio con parte real negativa.
\end{defi}

\begin{figure}
	\centering
	\includegraphics[width=1\textwidth]{img/EquilibriumPoints.pdf}
	\caption{Ejemplo de puntos de equilibrio en el origen de un sistema lineal bidimensional.}
	\label{fig: EquilibriumPoints}
\end{figure}

\begin{defi}
	Sean $E$ un subconjunto de $\Rn$, $f \in C^{1}(E)$ y $\phi_{t}$ el flujo del sistema \eqref{eq: sisAut} definida para todo $t \in \R$. Entonces un conjunto $S \subset E$ es llamado invariante con respecto al flujo $\phi_{t}$ si $\phi_{t} \subset S$ para todo $t\in \R$, y $S$ es llamado invariante positivo (o negativo) con respecto al flujo $\phi_{t}$ si $\phi_{t} \subset S$ para todo $t \geq 0$ (o $t\leq0$).
\end{defi}

\begin{teo}[Hartman-Grobman]\label{teo: hartman}
	Sea $E$ un subconjunto abierto de $\Rn$ que contiene al origen, sea $f \in C^{1}(E)$, y sea $\phi_{t}$ el flujo del sistema no lineal \eqref{eq: sisAut}. Suponga que $0$ es un punto de equilibrio hiperbólico. Entonces existe un homeomorfismo $H$ de un conjunto abierto $U$ que contiene al origen hacia un conjunto abierto $V$ que contiene al origen, tal que para cada $x_{0} \in U$, existe un intervalo abierto $I_{0} \subset \R$ que contiene al cero, tal que para todo $x_{0} \in U$ y $t \in I_{0}$
	
	\begin{equation*}
		H \circ \phi_{t}(x_{0}) = e^{At}H(x_{0});
	\end{equation*}
	
	es decir, H mapea trayectorias de \eqref{eq: sisAut} que estan cerca del origen, hacia trayectorias de \eqref{eq: sisLinAuto} cerca al origen preservando la parametrización por tiempo.
\end{teo}

Este importante teorema nos indica que el retrato fase cerca a un punto crítico hiperbólico es topológicamente equivalente al retrato fase de la linealización en dicho punto de equilibrio. Aquí la \textit{equivalencia topológica} es dada principalmente por la existencia de un \textit{homeomorfismo} (una deformación continua con inversa continua) que mapea trayectorias de un retrato fase local hacia el otro, preservando el sentido del tiempo. Intuitivamente, dos retratos fases son topológicamente equivalentes si una es una versión distorsionada de la otra.

Si
\begin{equation}
	\dot{x} = f_{\mu}(x) \quad x \in \Rn, \quad \mu \in \R^{k}
	\label{eq: sisParametro}
\end{equation}
es un sistema de ecuaciones diferenciales que depende del parámetro $k$-dimensional $\mu$, entonces los puntos de equilibrio de \eqref{eq: sisParametro} son dados por las soluciones de la ecuación $f_{\mu}(x) = 0$. Cuando $\mu$ varia, la estructura cualitativa del flujo puede cambiar, en particular, puntos críticos pueden ser creados o destruidos, o también su estabilidad puede cambiar, de igual manera trayectorias cerradas pueden aparecer o desaparecer. Estos cambios cualitativos en las dinámicas del sistema son llamadas \textit{bifurcaciones} y los valores paramétricos para los cuales este cambio ocurre son llamados \textit{valores de bifurcación}.